\documentclass{article}
\usepackage{graphicx}
\usepackage{xcolor}
\usepackage[hidelinks]{hyperref}
\begin{document}
\begin{center}
CLARION UNIVERSITY OF PENNSYLVANIA\\
COLLEGE OF BUSINESS ADMINISTRATION\\
DEPARTMENT OF FINANCE
\\
{\bf Investments}\\
{\bf FIN 376}\\
{\bf Fall 2015}\\
\end{center}
\vspace*{5pt}
{\bf Instructor}: Matthew Brigida, Ph.D. \\
{\bf Office}: Still Hall 318\\
{\bf Office Hours}:  In Still Hall Office:  Monday 2pm--5pm $|$ Tuesday and Thursday 3:15pm--4:15pm\\
{\bf Email}: 
\href{mailto:mbrigida@clarion.edu}{\textcolor{blue}{mbrigida@clarion.edu}} or \href{mailto:matt@complete-markets.com}{\textcolor{blue}{matt@complete-markets.com}} \\
\\
{\bf Class Location}:  Still Hall 208\\
{\bf Class Day \& Time}: Tuesday \& Thursday: 12:30-1:45pm
\\
{\bf Text}: {\it Investments} by Bodie, Kane, and Marcus, 8th edition, ISBN: 0077261453.
\\
\\
\begin{center}
{\bf DESCRIPTION}
\end{center}  
An introductory survey of the fundamental principles of investment management.
The learning outcomes for this course are summarized below:
\begin{enumerate}
\item  Understanding the structure of various capital markets, as well as how and why organizations
(ranging from individuals to multinational firms) participate in these markets. Particular importance will
be assigned to innovations in market structures (dark pools, ECNs) and their implications for market
participants. The student will be able to set up buy and sell orders for various securities, as well as
understand the use of leverage inherent in margin accounts and some derivative securities.
\item To extend previous study on equity and fixed income security valuation. Further, we will introduce
and value hybrid securities which have characteristics of both debt and equity (particularly convertible
bonds). Valuing the warrant in a convertible bond will also serve as a prelude to derivative securities.  
\item Quantify the interest rate risk in fixed income securities using duration and convexity. Moreover, the
student will be able to discuss the allocation of interest rate risk between lender and borrower in fixed
rate as compared to adjustable rate securities.
\item Assess reinvestment risk in fixed income securities and the benefits inherent in the option to prepay. 
\item This course will also introduce exchange­listed derivative securities (futures and options), along with
basic hedging and speculation strategies using these securities. Students will also learn about the
markets wherein derivatives are traded, and how laws governing these markets may differ from the laws
governing the stock markets. A basic introduction to valuing futures on storable commodities,
currencies, and interest rates will be afforded. We will then briefly discuss futures on non­storable commodities (particularly electricity). Lastly, using Excel students will apply the Black­Scholes (1973)
option pricing model to valuing options on non­dividend paying stock. The assumptions of
Black­Scholes (1973) will be discusses along with a non­rigorous introduction to risk neutral pricing
(intended to motivate further study).
\item Students will be able to measure the risk and return of individual securities as well as of a portfolio of
multiple risky securities. The effect of forming a portfolio on risk and expected return will be quantified,
as well as the resulting implications for forming an efficient portfolio. We will derive the Capital Asset
Pricing Model (CAPM) and thereby calculate the expected return for individual securities. A short
discussion of the assumptions of the CAPM and the alternative 'Arbitrage Pricing Theory' (APT) will
conclude.
\end{enumerate}
\begin{center}
{\bf ACADEMIC HONESTY POLICY}
\end{center} 
Academic dishonesty will not be tolerated in this class. Cheating
on quizzes, examinations, and other forms of dishonesty (e.g., plagiarism, collusion, and
falsification of data) will be dealt with in a serious and formal manner. The penalty for academic
dishonesty in this class will be course failure. That is, any student who is found to be cheating
or engaged in other academically dishonest behavior will be failed for this course for this
semester. Course withdrawals to avoid such a failure will not be permitted. As a student, you
have a responsibility to become familiar with the Academic Honesty Policy found in the {\it Student
Rights, Regulations, and Procedures Handbook}.\\
\\
\\
\begin{tabular}{|p{2.1 in}|p{3.1 in}|} \hline
\multicolumn{2}{|c|}{\bf BSBA Learning Goals and Objectives} \\ \hline
{\bf Goal or Objective} & {\bf Assessed by:} \\ \hline
Goal 1.0: Demonstrate Business Disciplinary Competence & The exams and homeworks will evaluate both equity and fixed income security valuation.\\ \hline
%
Goal 3.0 (Objectives 3.1 and 3.2): Communicate Effectively Orally and in Written Form &  The presentation of a student created Excel spreadsheet to  calculate stock option prices using the Black­Scholes (1973) model. \\ \hline
%
Goal 4.0 (Objectives 4.1 and 4.3): Demonstrate Analytical Thinking Skills & Students will learn to value securities by the principle of no­arbitrage. Further, students will discern which derivative securities may be valued by no­arbitrage and which may not \\ \hline
%
Goal 5.0: Understand Global Issues in the Functional Areas of Business &  New exchanges spanning of multiple continents (e.g. NYSE Euronext) will be discussed with particular attention being paid to their effect on the investment landscape. \\ \hline
%
Goal 6.0 (Objectives 6.1 and 6.3): Demonstrate Effective Use of Technology and Data Analysis &  In both homeworks and these project, students will value complex securities using Excel. \\ \hline
%
Objectives 1.1 (knowledge of a key business discipline), 4.1 (interpretation of evidence), 4.3 (formulation of conclusions), and 6.3 (understanding data analysis) &  Students will measure both individual asset and portfolio risk and return. Through analyzing the effect of portfolio construction on risk and return, students will derive the CAPM. After a discussion of the assumptions of the CAPM, students will weigh the model against the APT. \\ \hline
\end{tabular}
\vspace*{7pt}
\begin{center}
{\bf EXAMS} 
\end{center}
There will be three exams (two during the semester and a final exam). Normally no make-up exams will be given.  Failure to take an exam will result in a grade of zero for the missed exam.  Make-up exams will only be allowed for {\it extraordinary} and {\it verifiable} reasons.\\
\\
\begin{center}
{\bf HOMEWORK} 
\end{center}
\begin{itemize}
\item {\bf Written:} Three written homework assignments will be assigned during the semester (they will be
end-of-chapter problems from the text). The three homework assignments will be due the week before
each exam. Each homework will be worth 3 and 1/3 final grade points. Late homework will not be
accepted.
%
\item {\bf Computer/Trading Assignments/Possible Quizzes:} I'll assign several homeworks throughout the
semester that involve either trading in your brokerage accounts, or downloading data in R and
performing some calculation/analyses. If I give pop quizzes, these grades will be included here. The
sum of all these homeworks will be worth 15 final grade points ­ each individual homework will be
equally weighted.
\end{itemize}
\vspace*{5pt}
\begin{center}
{\bf BROKERAGE ACCOUNTS} 
\end{center}
We will use paper trading accounts provided by Interactive Brokers. These
trading accounts are the exact same as the actual brokerage accounts ­ except the money isn’t real.
You will have access to (and real data from) stock, bond, commodity, and foreign exchange markets.
While the data alone is worth a fair amount of money, Interactive Brokers is offering the accounts to us
for free. Keep this in mind while following the directions to set up your account ­ customer service will
be nonexistent. {\bf If you lose your password, or forget your username, etc, you will not have an
account for the semester.} In this case you will have to trade in another student’s account, or mine.
Please pay attention while setting up your account, and write everything down.
\vspace*{5pt}
\begin{center}
{\bf BLACK-SCHOLES PROJECT}
\end{center}
In groups, students will create a spreadsheet which will value an option on a non­dividend
paying stock using the Black­Scholes (1973) option pricing model. Further, the group must calculate
the stock's historical annualized volatility. To do so the group must show it is able to download a recent
time series of the underlying stock price, convert these prices into a time series of returns, calculate the
standard deviation of the returns, and then annualize the standard deviation (this is the stock annualized
volatility which is a parameter in the option pricing model). Each student in the group must be ready to
explain any part of the spreadsheet. In the event a student in a group cannot sufficiently answer
questions regarding the calculations, the student may receive a lower grade than the rest of the group. 
\vspace*{5pt}
\begin{center}
{\bf WEB APP PROJECT}
\end{center}
Students will create a Shiny interactive web application.  To do so you'll first need to sign up for a free \href{https://www.shinyapps.io/}{\textcolor{blue}{shinyapps}} account.  

You are free to create the account under a pseudonym, so no one can publicly identify you as the owner of the account.  However, the web application is a useful tool to show off your work, and is something that can go on your resume (with a link to the application).  So you may prefer to use your real name.  My user name is `mattbrigida'.  

Your application should have something to do with currency markets, and should be at least somewhat original.  See a gallery of applications here:  \href{http://shiny.rstudio.com/}{\textcolor{blue}{shiny.rstudio}}. Possible applications may be:
\begin{itemize}
\item Plot a time series of stock prices, returns, or volatility.
\item {\bf Financial Advisers:}  Create an app which will return target asset allocation given a person's age and investing goals/risk tolerance.
\item Create and plot a stock index.
\item A Black-Scholes calculator.
\item A margin calculator.
\item Create a histogram or probability density plot for bond or stock returns.
\end{itemize}
To get started you will want to use the RStudio development environment for R.  This is available in the Still hall computer lab, or you can install it for free on your own computer from here:  \href{https://www.rstudio.com/products/rstudio/download/}{\textcolor{blue}{download}}.  If you install it on your own computer you'll need to install R first.  You can get R here:  \href{https://cran.r-project.org/}{\textcolor{blue}{download}}
\vspace*{5pt}
\begin{center}
{\bf COURSE COMMUNICATION}
\end{center}
All important/official announcements will either be posted on Desire2Learn
or emailed to each student's Clarion University email account. I will post helpful information
to: \href{http://www.complete-markets.com}{\textcolor{green}{Complete Markets}}. To see information relating to your course type ``FIN 376'' in
the search bar at the upper left of the web page. Some examples of helpful information are
spreadsheets which assist in studying for exams or completing homeworks, answers to questions
other students have asked (of course I will not include who asked the question), and useful \href{http://www.r-project.org}{\textcolor{green}{R}} code. \\
\\
\\
\begin{center}
\begin{tabular}{lcr}
\multicolumn{3}{c}{\bf GRADING:} \\ \hline
Exam 1 & ................. & 20 \\

Exam 2 & ................. & 20 \\

Final Exam & ................. & 25\\

% Written Homework & ................. & 10 \\

Computer Homework & ................. & 15 \\

Black-Scholes Project & ................. &  10 \\

Web Application Project & ..........& 10 \\

Total Points & ................. & 100\\
\end{tabular}
\end{center}
%%
%%
\vspace*{5pt}
\begin{center}
{Final grades will be assigned according to the following scale}:
\end{center}
\begin{center}
\begin{tabular}{lr}
90 - 100 &  A \\
80 - 89.9 &  B \\
70 - 79.9 &  C \\
60 - 69.9 &  D \\
$<$ 60 &  F \\
\end{tabular}
\end{center}
%%
%%
\vspace*{5pt}
\begin{center}
{\bf An Important Note on Grading}
\end{center}
There is no special consideration if you need a certain grade in this course to graduate.  {\bf If you require a certain grade in this class to graduate it is your responsibility to earn that grade.} Specifically if you receive a `D' in this course I will not allow you to do extra assignments after the course is complete in exchange for a higher grade. 
%%
%%
\begin{center}
{\bf Adding and dropping this course}
\end{center}
The instructor is not involved in any way with your adding and dropping the course.  It is the student's responsibility to abide by all proper procedures and dates.  \\
\begin{center}
{\bf GENERAL NOTES}:
\end{center}
\begin{enumerate}
\item Attending class, and reading the text is required.
\item All exams will be closed book.
\item If you are late for an exam, no extra time will be allotted to you.
\item There will be no make up exams or extra points assignments.
\item You will be responsible for any material covered in class that is not in your text.
\item You should bring your text to class.
\item You are expected to be on time for class. This is especially important for exam
dates.
\item Disruptive behavior in the classroom will not be tolerated.
\item You may not use tobacco products in class.
\end{enumerate}
\begin{center}
\vspace*{5pt}
{\bf TENTATIVE OUTLINE}
\end{center}
\begin{itemize}
\item 8/24: Chapter 1
\item 8/31: Chapter 2 \& 3
\item 9/7: Chapter 3 \& 4
\item 9/14: Chapter 14
\item 9/22 (exam review) \& {\bf 9/24 (Exam 1)}
\item 9/31: Chapter 15
\item 10/5: Chapter 16
\item 10/12: Chapter 20
\item 10/19: Chapter 21
\item 10/27 (exam review) - {\bf 10/29 (Exam 2)}
\item 11/2: Chapter 21
\item 11/9: Chapter 22
\item 11/16: Chapter 23
\item 11/30: Trading, exam review, {\bf and the last day to present your projects.}
\item 12/7: Finals Week
\end{itemize}

\clearpage

\begin{center}
{\bf Statement Required by PASSHE}  
\end{center}

Clarion University and its faculty are committed to assuring a safe and productive educational environment for all students. In order to meet this commitment and to comply with Title IX of the Education Amendments of 1972 and guidance from  the Office for Civil Rights, the University requires faculty members to report incidents of sexual violence shared by students to the University's Title IX Coordinator.                         

The only exceptions to the faculty member's reporting obligation are when incidents of sexual violence are communicated by a student during a classroom discussion, in a writing assignment for a class, or as part of a University-approved research project.                           

Faculty members are obligated to report sexual violence or any other abuse of a student who was, or is, a child (a person under 18 years of age) when the abuse allegedly occurred to the person designated in the University protection of minors policy. 

\end{document}
